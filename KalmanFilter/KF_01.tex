\documentclass[UTF8,a4paper,10pt]{ctexart}
\usepackage[left=2.50cm, right=2.50cm, top=2.50cm, bottom=2.50cm]{geometry} %页边距
\CTEXsetup[format={\Large\bfseries}]{section} %设置章标题居左
 
 
%%%%%%%%%%%%%%%%%%%%%%%
% -- text font --
% compile using Xelatex
%%%%%%%%%%%%%%%%%%%%%%%
% -- 中文字体 --
%\setmainfont{Microsoft YaHei}  % 微软雅黑
%\setmainfont{YouYuan}  % 幼圆    
%\setmainfont{NSimSun}  % 新宋体
%\setmainfont{KaiTi}    % 楷体
%\setmainfont{SimSun}   % 宋体
%\setmainfont{SimHei}   % 黑体
% -- 英文字体 --
%\usepackage{times}
%\usepackage{mathpazo}
%\usepackage{fourier}
%\usepackage{charter}
\usepackage{helvet}
 
 
\usepackage{amsmath, amsfonts, amssymb} % math equations, symbols
\usepackage[english]{babel}
\usepackage{color}      % color content
\usepackage{graphicx}   % import figures
\usepackage{url}        % hyperlinks
\usepackage{bm}         % bold type for equations
\usepackage{multirow}
\usepackage{booktabs}
\usepackage{epstopdf}
\usepackage{epsfig}
\usepackage{algorithm}
\usepackage{algorithmic}
\renewcommand{\algorithmicrequire}{ \textbf{Input:}}     % use Input in the format of Algorithm  
\renewcommand{\algorithmicensure}{ \textbf{Initialize:}} % use Initialize in the format of Algorithm  
\renewcommand{\algorithmicreturn}{ \textbf{Output:}}     % use Output in the format of Algorithm  
\usepackage{CJKulem}


 
\usepackage{fancyhdr} %设置页眉、页脚
%\pagestyle{fancy}
\lhead{}
\chead{}
%\rhead{\includegraphics[width=1.2cm]{fig/ZJU_BLUE.eps}}
\lfoot{}
\cfoot{}
\rfoot{}
 

 
\usepackage{hyperref} 
\hypersetup{colorlinks, bookmarks, unicode} 
 
\begin{document}
\subparagraph{1. 问题描述}
    现有系统模型如下:
    \begin{equation}
        \left\{
            \begin{array}{cc}
               x_k = Ax_{k-1}+Bu_{k-1} + w_{k-1},\\ \\
               y_k = Cx_{k} + v_{k}
            \end{array}
        \right.
    \end{equation}\label{model}

其中,列向量$x$表示系统内部状态变量,$A$表示系统状态转移矩阵,$B$表示控制输入矩阵,$y$表示测量值,
$w$和$v$表示均为过程噪声和测量噪声列向量,满足正态分布且协方差矩阵分别为$Q$,$R$.

如果不考虑噪声$w$或者$v$,我们可以得到系统$k$时刻的估计值值为
\begin{equation}
    \hat{x}_{\bar{k}} = A\hat{x}_{k-1}+Bu_{k-1} \quad \text{or} \quad  \hat{x}_{k_{m}} = C^{-1}y_k
\end{equation}\label{predict_forward}
目前来说,我们手上有的数据是一个受噪声影响的先验估计值$\hat{x}_{\bar{k}}$和一个受噪声影响的测量值 $\hat{x}_{k_{m}}$。
问题来了,我们应该\uline{如何使用这两个不太准确的值来得出一个相对精确的后验估计$\hat{x}_{k}$}?.

\subparagraph{2. 数据融合}
    根据滤波思想中的线性加权组合数据,可以尝试以下方案
    \begin{equation}
        \hat{x}_{k} = \hat{x}_{\bar{k}} + G(\hat{x}_{k_{m}}- \hat{x}_{\bar{k}})
    \end{equation}\label{predict_r}

    显然,上述思路中描述的是,给状态变量的先验估计和测量值分别给不同的权重,再取和。简单的用$K_k=GC^{-1}$做替换可以得到
    \begin{equation}
        \hat{x}_{k} = \hat{x}_{\bar{k}} + K_k(y_k- C\hat{x}_{\bar{k}})\label{predict}
    \end{equation}
    
    到目前为止,我们使用系统的上一时刻$k-1$的估计状态$\hat{x}_{k-1}$和系统输入计算出了当前时刻$k$的状态的先验估计值$\hat{x}_{\bar{k}}$,
    手上还有的数据是当前时刻$k$的测量值$y_k$,结合两者设计了一个数据融合算法(4),
    当前问题在于,\uline{如何选择合适的增益$K_k$(称之为卡尔曼增益),来使得系统当前时刻$k$的状态的后验估计值更接近真实值$x_k$。}

\subparagraph{3. 卡尔曼增益设计}
    我们设计卡尔曼增益的目的是使系统$k$时刻的估计值$\hat{x}_k$尽可能向真实值$x_k$靠拢,在这里,我们要用符号$e_k$来表示两者之间的误差:
    \begin{equation}
        e_k = x_k - \hat{x}_k
    \end{equation}\label{e_k}
    可以简单推导出,$e_k=[e_1,e_2,...,e_n]^T$也该是一个正态分布的列向量,且用符号$P$表示其协方差矩阵,可以知道,
    \begin{equation}
        P_k = E[e^T_k e_k] = E[(x_k - \hat{x}_k)^T(x_k - \hat{x}_k)]
    \end{equation}\label{sdsf}
    
    考虑一下,怎样才能算是说系统$k$时刻的估计值$\hat{x}_k$向真实值$x_k$靠拢的很好呢,也就是说这个估计怎样才能算最优呢?
    \uline{通过计算估计误差的协方差矩阵的迹$tr(M)$,$tr(M)$越小,估计误差的协方差的迹越小,我们认为此时系统的估计最优。}
    这一块估计最优的来源需要重新理解,两个正态分布数据融合?贝叶斯滤波?\\    
    1. \href{https://blog.csdn.net/varyshare/article/details/97642209}{贝叶斯滤波----为什么是两个高斯分布相乘相乘}\\
    2. \href{https://blog.csdn.net/michaelhan3/article/details/85453368}{高斯分布相乘仍旧是高斯分布,且均值为*}\\
    3. \href{https://zhuanlan.zhihu.com/p/105496881}{最小化方差,极大似然估计的联系}\\

    把(4)代入(6)中,可以得到
    \begin{equation}
        \begin{aligned}
            P_k &= E[(x_k - \hat{x}_k)(x_k - \hat{x}_k)^T]\\
              &= E[(x_k - \hat{x}_{\bar{k}} - K_k(y_k- C\hat{x}_{\bar{k}})) (x_k - \hat{x}_{\bar{k}} - K_k(y_k- C\hat{x}_{\bar{k}}))^T]\\
              &= E[(x_k - \hat{x}_{\bar{k}} - K_k(Cx_{k} + v_{k}- C\hat{x}_{\bar{k}})) (x_k - \hat{x}_{\bar{k}} - K_k(Cx_{k} + v_{k}- C\hat{x}_{\bar{k}}))^T]\\
            %   &= E[(x_k - \hat{x}_{\bar{k}} - K_k(Cx_{k} + v_{k}- C\hat{x}_{\bar{k}})) (x_k - \hat{x}_{\bar{k}} - K_k(Cx_{k} + v_{k}- C\hat{x}_{\bar{k}}))^T]\\
        \end{aligned}
    \end{equation}
    定义先验估计误差$e_{\bar{k}} = x_k - \hat{x}_{\bar{k}}$且用$P_{\bar{k}}$来表示先验估计误差的协方差,则有
    \begin{equation}
        \begin{aligned}
            P_k  & = E[(e_{\bar{k}} - K_k(v_k + Ce_{\bar{k}}) (e_{\bar{k}} - K_k(v_k- Ce_{\bar{k}})^T]\\
                  & = E[((I-K_kC)e_{\bar{k}} -K_kv_k) ((I-K_kC)e_{\bar{k}} -K_kv_k)^T]\\
                  & = E[(I-K_kC)e_{\bar{k}}e^T_{\bar{k}}(I-K_kC)^T +  K_kv_kv^T_kK^T_k]
                    + E[(I-K_kC)e_{\bar{k}}v^T_k K^T_k] + E[K_kv_ke^T_{\bar{k}}(I-K_kC)]\\
                  & = E[(I-K_kC)e_{\bar{k}}e^T_{\bar{k}}(I-K_kC)^T +  K_kv_kv^T_kK^T_k]\\
                  & = (I-K_kC)P_{\bar{k}}(I-K_kC)^T + K_kRK^T_k
        \end{aligned}
    \end{equation}
    附注:其中第三个等式到第四个等式用到了
    $E[(I-K_kC)e_{\bar{k}}v^T_k K^T_k] = (I-K_kC)E [e_{\bar{k}}]E[v^T_k]K^T_k = 0$,
    相似地,有$E[K_kv_ke^T_{\bar{k}}(I-K_kC)]=0$

    在(8)的基础上继续推导,可以得到
    \begin{equation}
        \begin{aligned}
            tr[P_k] & = tr[(I-K_kC)P_{\bar{k}}(I-K_kC)^T + K_kRK^T_k]\\
                  & = tr [(P_{\bar{k}} -K_kCP_{\bar{k}})(I-C^TK^T_k) + K_kRK^T_k]\\
                  & = tr [P_{\bar{k}} -K_kCP_{\bar{k}} - P_{\bar{k}}C^TK^T_k + K_kCP_{\bar{k}}C^TK^T_k+K_kRK^T_k ]\\
                  & = tr [P_{\bar{k}} -2K_kCP_{\bar{k}} + K_kCP_{\bar{k}}C^TK^T_k+K_kRK^T_k ]
        \end{aligned}
    \end{equation}

    化简到(9),则下一步就是对卡尔曼增益$K_k$求导,求出极值点,则认为该点处的估计误差的协方差的迹$tr(P_k)$最小,求导并使之为0如下:
    \begin{equation}
            \frac{d(tr(P_k))}{d(K_k)} = -2P^T_{\bar{k}}C + 2K_kCP_{\bar{k}}C^T+ 2K_kR=0
    \end{equation}
    上述推导用到如下公式:
    \begin{equation}\nonumber
        \begin{aligned}
            &\frac{d(tr(AB)}{dA} = B^T\\
            &\frac{d(tr(ABA^T))}{dA} = A(B+B^T)
        \end{aligned}
    \end{equation}
    由(10)可以得到,$-P^T_{\bar{k}}C + K_k(CP_{\bar{k}}C^T+ R)=0$,即,
    \begin{equation}
        K_k = \frac{P^T_{\bar{k}}C}{CP_{\bar{k}}C^T+ R}
    \end{equation}

    到目前为止,我们使用$k$时刻先验估计协方差矩阵$P_{\bar{k}}$计算出$k$时刻的卡尔曼增益$K_k$,使得数据融合算法(4)得到的系统状态估计值$\hat{x}_k$是最优估计,
    (最优这一说法还未考量);
    而$k$时刻先验估计协方差矩阵$P_{\bar{k}}$的计算方法还为确定。

    \subparagraph{4. 先验估计协方差矩阵$P_{\bar{k}}$推导}
        根据先验估计协方差矩阵$P_{\bar{k}}$的定义,可以知道
        \begin{equation}
            \begin{aligned}
                P_{\bar{k}} &= E[e_{\bar{k}}e^T_{\bar{k}}]\\
                            &= E[(x_k - \hat{x}_{\bar{k}}) (x_k - \hat{x}_{\bar{k}})^T]\\
            \end{aligned}
        \end{equation}
    而根据系统模型(1),和先验估计的定义(2),可知
    \begin{equation}
        \begin{aligned}
            x_k - \hat{x}_{\bar{k}} &= Ax_{k-1}+Bu_{k-1} + w_{k-1} - (A\hat{x}_{k-1} + Bu_{k-1})\\
                                    &= A(x_{k-1} -\hat{x}_{k-1}) + w_{k-1}\\
                                    &= Ae_{k-1}+ w_{k-1}\\
        \end{aligned}
    \end{equation}
    上述公式的口语翻译是:$k$时刻的先验估计误差等于$k-1$时刻的估计误差与A的乘积再加上过程噪声。
    把(13)代入(12)可得:
    \begin{equation}
        \begin{aligned}
            P_{\bar{k}} &= E[(x_k - \hat{x}_{\bar{k}}) (x_k - \hat{x}_{\bar{k}})^T]\\
                        &= E[(Ae_{k-1}+ w_{k-1}))(e^T_{k-1}A^T+ w^T_{k-1})]\\
                        &= E[Ae_{k-1}e^T_{k-1}A^T] + E[w_{k-1}w_{k-1}^T]\\
                        &= AP_{k-1}A^T + Q
        \end{aligned}
    \end{equation}
    上述推导结果,即为所需要求的先验估计协方差矩阵$P_{\bar{k}}$的计算方法。

    \subparagraph{卡尔曼滤波过程}
    基本的推导已经完成,下面给出卡尔曼滤波的全过程。
    初始条件,给定$k-1$时刻后验估计值$\hat{x}_{k-1}$,控制输入$u_{k-1}$,估计误差协方差$P_{k-1}$.

    步骤一:计算先验估计值$\hat{x}_{\bar{k}}$
    \begin{equation}
        \hat{x}_{\bar{k}} = A\hat{x}_{k-1} + Bu_{k-1}
    \end{equation}

    步骤二:计算先验误差协方差$P_{\bar{k}}$
    \begin{equation}
        P_{\bar{k}} = AP_{k-1}A^T + Q
    \end{equation}

    步骤三:计算卡尔曼增益$K_k$
    \begin{equation}
        K_k = \frac{P^T_{\bar{k}}C^T}{CP_{\bar{k}}C^T+ R}
    \end{equation}

    步骤四:计算后验估计值$\hat{x}_k$
    \begin{equation}
        \hat{x}_{k} = \hat{x}_{\bar{k}} + K_k(y_k- C\hat{x}_{\bar{k}})
    \end{equation}

    步骤五:更新误差协方差
    \begin{equation}
        P_k = (I - K_kC)P_{\bar{k}}
    \end{equation}
    
    补充步骤五的推导:把(11)代入(8)即可得到
\end{document}